\documentclass{report}

\usepackage[a4paper]{geometry}

\usepackage{graphicx}
\usepackage{amsmath}
\usepackage{bm,nicefrac}
\usepackage[plain]{algorithm}
\usepackage[noend]{algpseudocode}
\usepackage{algpseudocode}
%\usepackage{amsthm}
\usepackage[dvipsnames]{xcolor}
\usepackage{array, makecell}
\usepackage[draft]{todonotes}
\usepackage{amssymb}
\usepackage{booktabs,caption}
\usepackage[flushleft]{threeparttable}

% Hyperref
\usepackage{hyperref}
\renewcommand{\chapterautorefname}{Chapter}
\newcommand{\algorithmautorefname}{Algorithm}

% UTF8 decoding
\usepackage[utf8]{inputenc}
\usepackage[T1]{fontenc}

% Biblatex
\usepackage[backend=biber,style=ieee,citestyle=authoryear]{biblatex}
\addbibresource{references.bib}

% Table
\newcolumntype{P}[1]{>{\centering\arraybackslash}p{#1}}

% Subplotting
\usepackage{subfig}

% Rotated table header
\usepackage{adjustbox}
\usepackage{array}

\newcolumntype{R}[2]{%
    >{\adjustbox{angle=#1,lap=\width-(#2)}\bgroup}%
    l%
    <{\egroup}%
}
\newcommand*\rot{\multicolumn{1}{R{45}{1em}}}

% Definition
%\theoremstyle{definition}
\newtheorem{definition}{Definition}[section]

%% TIkz
\usepackage{tikz}
\usepackage{verbatim}
\usetikzlibrary{calc,trees,positioning,arrows,chains,shapes.geometric,%
    decorations.pathreplacing,decorations.pathmorphing,shapes,%
    matrix,shapes.symbols, plotmarks}
\usepackage{tikz-uml}

    \newcommand*\diff{\mathop{}\!\mathrm{d}}
    
    % Sub sub sub section
\newcommand{\subsubsubsection}[1]{\paragraph{#1}\mbox{}\\}
\setcounter{secnumdepth}{4}
\setcounter{tocdepth}{4}

\usepackage[table]{colortbl}

\usepackage[titletoc]{appendix}


% References in content

\usepackage[nottoc]{tocbibind}

%% Glossaries
    
\usepackage[acronym,automake,nonumberlist]{glossaries}
\usepackage[automake]{glossaries}

%% MATLAB list

\usepackage{listings}

\usepackage[numbered,framed]{matlab-prettifier}

% Abbreviations
\newacronym{hjb}{HJB}{Hamilton-Jacobi-Bellman equation}
\newacronym{vd}{VD}{Voronoi diagram}
\newacronym{cd}{CD}{Cell decomposition}
\newacronym{vg}{VG}{Visibility graph}
\newacronym{pr}{PR}{Probabilistic roadmap}
\newacronym{rrt}{RRT}{Rapidly-exploring random tree}
\newacronym{pp}{PP}{Pontryagin's principle}
\newacronym{dp}{DP}{Dynamic programming}
\newacronym{dda}{DDA}{Direct discretized approximation}
\newacronym{dhjb}{DHJB}{Discretized Hamilton-Jacobi-Bellman}
\newacronym{ps}{PS}{Pseudospectral method}
\newacronym{gnc}{GNC}{Guidance, navigation, control}
\newacronym{ocp}{OCP}{Optimal control problem}
\newacronym{dof}{DOF}{Degree of freedom}

% Nomenclature
\newglossaryentry{route}{
  name = route,
  description =   {Sequence of waypoints on a map that the ship should follow in order to reach a goal},
}
\newglossaryentry{digital_twin}{
  name = digital twin,
  description =   {Real time digital replica of a physical device},
}
\newglossaryentry{eta}{
  name = $\boldsymbol{\eta}$,
  description =   {Ship state $[x,y,\psi]^T$},
}
\newglossaryentry{x}{
  name = $x$ ,
  description = North position in NED frame,
}
\newglossaryentry{y}{
  name = $y$ ,
  description = East position in NED frame,
}
\newglossaryentry{psi}{
  name = $\psi$ ,
  description = Yaw angle,
}
\newglossaryentry{u}{
  name = $u$ ,
  description = Surge linear velocity,
}
\newglossaryentry{u_d}{
  name = $u_d$ ,
  description = Desired surge linear velocity,
}
\newglossaryentry{v}{
  name = $v$ ,
  description = Sway linear velocity,
}
\newglossaryentry{r}{
  name = $r$ ,
  description = Yaw angular velocity,
}
\newglossaryentry{U}{
  name = $U$ ,
  description = Linear speed,
}
\newglossaryentry{U_d}{
  name = $U_d$ ,
  description = Desired linear speed,
}
\newglossaryentry{beta}{
  name = $\beta$ ,
  description = Sideslip angle,
}
\newglossaryentry{e}{
  name = $e$ ,
  description = Cross-track error,
}
\newglossaryentry{s}{
  name = $s$ ,
  description = Along-track error,
}
\newglossaryentry{tau}{
  name = $\tau$ ,
  description = {Control input vector},
}
\newglossaryentry{X}{
  name = $X$ ,
  description = Surge force,
}
\newglossaryentry{Y}{
  name = $Y$ ,
  description = Sway force,
}
\newglossaryentry{N}{
  name = $N$ ,
  description = Yaw moment,
}



\glsunsetall
\makeglossaries

